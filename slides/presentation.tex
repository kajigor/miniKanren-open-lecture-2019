\documentclass{beamer}
\usepackage{beamerthemesplit}
\usepackage{wrapfig}
\usetheme{SPbGU}
\usepackage{pdfpages}
\usepackage{amsmath}
\usepackage{cmap} 
\usepackage[T2A]{fontenc} 
\usepackage[utf8]{inputenc}
\usepackage[english,russian]{babel}
\usepackage{indentfirst}
\usepackage{amsmath}
\usepackage{tikz}
\usepackage{multirow}
\usepackage[noend]{algpseudocode}
\usepackage{algorithm}
\usepackage{algorithmicx}
\usepackage{listings}
\usepackage{pifont}% http://ctan.org/pkg/pifont
\usepackage{xcolor}
\usepackage{mdframed}
\usepackage{multicol}
\newcommand{\cmark}{\ding{51}}%
\newcommand{\xmark}{\ding{55}}%
\newcommand{\miniKanren}{\texttt{miniKanren}}
\newcommand{\ocaml}{\texttt{OCaml}}
\newcommand{\mediKanren}{\texttt{mediKanren}}

\usepackage{epstopdf}
\usepackage{forest}
\usetikzlibrary{shapes,arrows}
\usepackage{fancyvrb}
\newtheorem{rutheorem}{Theorem}
\beamertemplatenavigationsymbolsempty
\setbeamertemplate{itemize items}[circle]

\lstdefinelanguage{ocaml}{
%keywords={fresh, let, begin, end, in, match, type, and, fun, function, try, with, when, class, 
%object, method, of, rec, %repeat, 
%until, while, not, do, done, as, val, inherit, 
%new, module, sig, deriving, datatype, struct, if, then, else, open, private, virtual, ostap},
sensitive=true,
basicstyle=\small,
commentstyle=\small\itshape\ttfamily,
keywordstyle=\ttfamily\underbar,
identifierstyle=\ttfamily,
basewidth={0.5em,0.5em},
columns=fixed,
fontadjust=true,
literate={->}{{$\;\;\to\;\;$}}1
         {===}{{$\equiv$}}1
         {&&&}{{$\wedge$}}1
         {|||}{{$\vee$}}1
         {fresh}{{$\exists$}}1,
morecomment=[s]{(*}{*)}
}

\lstset{
basicstyle=\small,
identifierstyle=\ttfamily,
keywordstyle=\bfseries,
commentstyle=\scriptsize\rmfamily,
basewidth={0.5em,0.5em},
fontadjust=true,
%escapechar=~,
language=ocaml,
mathescape=true,
moredelim=[is][\bfseries]{[*}{*]}
}

\definecolor{SadRed}{RGB}{255, 180, 180}
\definecolor{MehYellow}{RGB}{255, 255, 180}
\definecolor{HappyGreen}{RGB}{180, 255, 180}

\newmdenv[backgroundcolor=SadRed, innertopmargin=8,innerbottommargin=1, linecolor=SadRed]{badcode}

\newmdenv[backgroundcolor=MehYellow, innertopmargin=8,innerbottommargin=1, linecolor=MehYellow]{mehcode}

\newmdenv[backgroundcolor=HappyGreen, innertopmargin=8,innerbottommargin=1, linecolor=HappyGreen]{goodcode}

\title[]{Реляционное программирование}
\institute[]{
Лаборатория языковых инструментов JetBrains
}

\author[Екатерина Вербицкая]{Екатерина Вербицкая}

\date{13 декабря 2019}

\definecolor{orange}{RGB}{179,36,31}

\begin{document}
{

\begin{frame}
      \begin{center} 
        {\includegraphics[width=1.5cm]{pics/jb.png}} 
      \end{center}

  \titlepage
\end{frame}
}

\begin{frame}[fragile]
  \transwipe[direction=90]
  \frametitle{Реляционное программирование}
  
  \begin{center}
  Программа --- отношение 
  \end{center}
  
\begin{center}
$append^{o} \subseteq [A] \times [A] \times [A] $ 

$$
\begin{array}{lrcl}
append^{o} & = & \{ & ([\,], [\,], [\,]); \\
           &   &    & ([0], [\,], [0]); \\ 
           &   &    & ([1], [\,], [1]); \\
           &   &    & \dots \\
           &   &    & ([\,], [0], [0]); \\
           &   &    & \dots \\
           &   &    & ([4], [2], [4, 2]); \\
           &   &    & \dots \\
           &   &    & ([4, 2], [13], [4, 2, 13]); \\
           &   &    & \dots \\
           &   & \}
\end{array}
$$ 
\end{center}
\end{frame}

\begin{frame}[fragile]
  \transwipe[direction=90]
  \frametitle{Пример программы на \miniKanren}
\begin{center}
\begin{minipage}{4.5cm}
\begin{lstlisting}[frame=single]  
appendo x y z = 
  (x === [] &&& z === y)
  ||| (fresh h t r
       ( x === h : t 
       &&& z === h : r 
       &&& appendo t y r))
\end{lstlisting}
\end{minipage}
\end{center}
\end{frame}


\begin{frame}[fragile]
  \transwipe[direction=90]
  \frametitle{Вычисление в реляционном программировании}
  
$$ 
\begin{array}{lccccccl}
append^o & [1] & [2, 3] & q         &\rightarrow& \{ &  [1, 2, 3] & \} \\
append^o & [1] & q      & [1, 2, 3] &\rightarrow& \{ & [2, 3] & \} \\
append^o & q   & [1]    & [2, 3]    &\rightarrow& \{ & & \}  \\
\end{array}
$$

\end{frame}

\begin{frame}[fragile]
  \transwipe[direction=90]
  \frametitle{Вычисление в реляционном программировании}

$$ 
\begin{array}{rccl}
append^o \, q \, p \, [1, 2] &\rightarrow& \{ &                \\
                    &           &    & ([\,], [1,2]), \\
                    &           &    & ([1], [2]), \\
                    &           &    & ([1,2], [\,]) \\
                    &           & \} \\
                    &           & \\ 
append^o \, q \, q \, [2, 4, 2, 4] &\rightarrow& \{ &  [2, 4] \quad \} \\
                    &           & \\ 
append^o \, q \, p \, r      &\rightarrow& \{ &  \\
                    &           &    &([\,], \ \alpha, \ \alpha), \\
                    &           &    & ([\alpha], \  \beta, \ (\alpha:\beta)) \\
                    &           &    & ((\alpha : \beta),\  \gamma, \ (\alpha:(\beta:\gamma))) \\
                    &           &    & \dots \\
                    &           &    \}  \\
\end{array}
$$

\end{frame}


\begin{frame}[fragile]
  \transwipe[direction=90]
  \frametitle{Направление вычислений}

\begin{center}
  $foo^o \subseteq A \times B$
\end{center}

\begin{itemize}
  \item $foo^o \ \alpha \ q : A \rightarrow [B]$
  \item $foo^o \ q \ \beta  : B \rightarrow [A]$ --- в ``обратном'' направлении
  \item $foo^o \ q \ p      : () \rightarrow [(A \times B)] $ 
\end{itemize}  
 
\end{frame}

\begin{frame}[fragile]
  \transwipe[direction=90]
  \frametitle{OCanren}

\begin{itemize} 
  \item Строго типизированный \miniKanren
  \item Встроен в \ocaml
  \item Синтаксическое расширение для описания программ 
\end{itemize}
 
 \vfill
 
 \begin{center}
 Демонстрация
 \end{center}
\end{frame}

\begin{frame}[fragile]
  \transwipe[direction=90]
  \frametitle{Области применения}

\begin{itemize} 
  \item Синтез кода (обращение интерпретаторов, Barliman)
  \item Поиск в базах знаний (\mediKanren)
  \item Генерация тестовых сценариев
  \item Статический анализ кода 
  \item $\dots$
\end{itemize}
 
\end{frame}


\begin{frame}[fragile]
  \transwipe[direction=90]
  \frametitle{Синтез кода}

\begin{itemize} 
  \item Реализуем интерпретатор на функциональном языке
  \item Транслируем интерпретатор на \miniKanren
  \item Задаем частично определенную входную программу и ее результат выполнения
  \item Запускаем вычисление на \miniKanren
  \item \miniKanren \ должен заполнить ``дырки'' в программе
\end{itemize}
 
 
 \vfill
 \begin{center} 
 Демонстрация
 \end{center}
\end{frame}

\begin{frame}[fragile]
  \transwipe[direction=90]
  \frametitle{Как все это работает: демонстрация}
\end{frame}

\begin{frame}[fragile]
  \transwipe[direction=90]
  \frametitle{Трудности и открытые проблемы}
  \begin{itemize}
    \item Несмотря на полноту языка, завершения выполнения иногда можно не дождаться
    \begin{itemize}
      \item Если запросить больше ответов, чем существует 
      \item Если пространство поиска слишком большое 
    \end{itemize}
    \item Чувствительность к порядку конъюнктов
    \begin{itemize}
      \item Как определить, что конъюнкты надо пересортировать? 
    \end{itemize}
    \item Как определять, что попали в бесконечную ветвь? 
    \begin{itemize}
      \item В общем случае задача неразрешима
      \item Иногда помогает критерий Розплохоса 
    \end{itemize}
    \item При обращении интерпретаторов (синтез програм) пространство поиска \textbf{как правило} большое
    \begin{itemize}
      \item Специализация
    \end{itemize}
  \end{itemize}
  
\end{frame}

\end{document}
